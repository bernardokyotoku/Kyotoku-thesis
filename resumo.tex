Tomografia por coerência óptica (OCT) é uma técnica de imageamento não invasiva que usa radiação infravermelho para sondar alguns milímetros the profundidade de um alvo com um resolução de poucos micrômetros. Aqui, nós expomos a base teórica para entender a técnica. O texto cobre as duas variedades de OCT --- domínio temporal e domínio da frequência --- e descreve três aplicações da técnica em odontologia: a) Um na avalição the propagação rachaduras em polímeros reforçado com fibra usado em restauração dental; b) O imageamento da sobra de dentina e cavidade pulpar após excavação da dentina, com o propósito de medir a espessura da dentina, e c) uma avaliação clínica da integridade de restaurações dentais. Em todas essa aplicações, OCT gerou imagens marcantes e forneceu informações semiquatitativas sobre a estrura dentária.

Com o objetivo de desenvolver um sistema de tomografia óptica integrada em um chip. Nós expomos a base teórica da plataforma de fotônica integrada. Após uma revisão literária, nós descobrimos que não existe espectrômetro integrado com a especificações necessárias para uso em OCT. Nós, então, desenvolvemos um espectrômetro com a características necessárias. Isso foi possível devido a uma nova arquitetura de espectrômetro baseada na combinação de um ressoador em anel e um espectrômetro de grade de difração.
