%% LyX 1.6.5 created this file.  For more info, see http://www.lyx.org/.
%% Do not edit unless you really know what you are doing.
\documentclass[english]{article}
\usepackage[T1]{fontenc}
\usepackage[latin9]{inputenc}
\usepackage{array}
\usepackage{amstext}
\usepackage{esint}

\makeatletter

%%%%%%%%%%%%%%%%%%%%%%%%%%%%%% LyX specific LaTeX commands.
%% Because html converters don't know tabularnewline
\providecommand{\tabularnewline}{\\}

\makeatother

\usepackage{babel}

\begin{document}
%
\begin{table}


\centering{}\begin{tabular}{>{\centering}m{0.4\textwidth}ccccc}
\hline 
\medskip Reference & Diagonal & Resolution & FSR & Crosstalk & Diagona X Resolution\tabularnewline
 & (mm)\medskip  & (nm) & (nm) & (dB) & \tabularnewline
\hline
\hline 
Brouckaert & 0.3 & 20 & 150 & -30 & 6\tabularnewline
Cheben & 11 & 0.2 & 10 & -10 & 2.2\tabularnewline
Horst & 0.3 & 3.2 & 75 & -19 & 0.9\tabularnewline
RES & 1.4 & 0.1 & 115 & -8 & 0.14\tabularnewline
\hline
\end{tabular}\caption{}

\end{table}


%
\begin{table}
\noindent \centering{}\begin{tabular}{>{\centering}m{0.4\textwidth}cc>{\centering}m{0.3\textwidth}}
\hline 
 & $f\left(x\right)$ & $F\left(x\right)=\int_{-\infty}^{\infty}f\left(x\right)e^{-2\pi ix\xi}dx$ & \medskip Remarks\medskip \tabularnewline
\hline
\hline 
Rectangular $\leftrightarrow$ Sinc & $\text{rect}\left(ax\right)$ & ${\displaystyle \frac{1}{|a|}\cdot\text{sinc}\left(\frac{\xi}{a}\right)}$ & $\text{sinc}\left(x\right)=\frac{\sin\left(\pi x\right)}{\left(\pi x\right)}$\tabularnewline
Gaussian $\leftrightarrow$ Gaussian & ${\displaystyle e^{-\alpha x^{2}}}$ & ${\displaystyle \sqrt{\frac{\pi}{\alpha}}\cdot e^{-\frac{(\pi\xi)^{2}}{\alpha}}}$ & Gaussian function is its own Fourier tranform. $a>0$\tabularnewline
Exponential $\leftrightarrow$ Lorentzian & $e^{-a|x|}$ & ${\displaystyle \frac{2a}{a^{2}+4\pi^{2}\xi^{2}}}$ & $a>0$\tabularnewline
Exponential pulse $\leftrightarrow$  & ${\displaystyle e^{-ax}H(x)}$ & ${\displaystyle \frac{1}{a+2\pi i\xi}}$ & $H\left(x\right)$ is the Heaviside unit step funcion. $\Re\left(a\right)>0$\tabularnewline
\hline
\end{tabular}\caption{}

\end{table}

\end{document}
