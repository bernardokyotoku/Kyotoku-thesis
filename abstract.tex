Optical coherence tomography is a non-invasive imaging technique that uses non-ionizing infrared radiation to probe few millimeter depth of target at resolution of few micrometers. Here we expose the theoretical basis to understand the technique. The text covers the two varieties of OCT --- time domain and frequency domain--- and discribe three applications of this technique to dentistry: a) One in the evaluation of crack propagation in fiber reinforced polymers used for dental restoration; b) The imaging of remains dentin and pulp chamber after dentin excavation for the purpose of measurement of dentin thickness, and c) a clinical evaluation of the integrity of dental restoration. In all these applications, OCT has outstanding imaging results and provides semiquatitative insight into the dental structure.

With the aim of developing optical coherence tomography integrated to a chip, we expose the theoretical basis of integrated photonics platform. After literature review, we detected that no integrated spectrometer, and OCT component, with the needed specifications exists. We, then, developed a spectrometer with the necessary features. This was possible due the creation of novel spectrometer architecture based on the combination of a ring resonator and a diffraction grating spectrometer.
